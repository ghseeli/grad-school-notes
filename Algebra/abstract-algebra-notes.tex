\documentclass[11pt,leqno,oneside]{amsart}
%\usepackage{amsmath,amsthm}
\usepackage{amssymb,mathrsfs}
%\usepackage{ytableau}
%\ytableausetup{smalltableaux,centertableaux}
\usepackage{tikz}
\usepackage{upgreek}
\usepackage{enumitem}
%\usepackage[nohead,nofoot,centering]{geometry}

\usepackage{color} 
%% Some user-defined colors
      \definecolor{mydefi}{cmyk}{1,0,0,.5}
      \definecolor{myred}{rgb}{.7,.1,.1}
      \definecolor{myblue}{rgb}{.1,.1,.6}
      \definecolor{mygreen}{rgb}{.1,.6,.1}

\usepackage[urlbordercolor={1 1 1}, pdfborder={0 0 0}, bookmarks=true,
  colorlinks=true, linkcolor=myblue, citecolor=myblue,
  urlcolor=myblue, hyperfootnotes=false]{hyperref}

\usepackage[alphabetic,abbrev]{amsrefs} % use AMS ref scheme

\addtolength{\footskip}{2\baselineskip} % to lower the page numbers


%%%%%%%%%%%%%%%%%%%%%%%%%%%%%%%%%%%%%%%%%%%%%%%%%%%%%%%%%%%%%%%%%%%
%%  MACRO DEFINITIONS:  Co-authors -- PLEASE use these! 
%%%%%%%%%%%%%%%%%%%%%%%%%%%%%%%%%%%%%%%%%%%%%%%%%%%%%%%%%%%%%%%%%%%
\newcommand{\N}{{\mathbb N}} % natural numbers
\newcommand{\Z}{{\mathbb Z}} % integers
\newcommand{\Q}{{\mathbb Q}} % rational numbers
\newcommand{\R}{{\mathbb R}} % real numbers
\newcommand{\C}{{\mathbb C}} % complex numbers
\newcommand{\End}{\operatorname{End}} % endomorphisms
\newcommand{\Hom}{\operatorname{Hom}} % homomorphisms
\newcommand{\GL}{\operatorname{GL}} % general linear group
\newcommand{\B}{\mathfrak{B}} % use for the Brauer algebra
\newcommand{\Sym}{\mathfrak{S}} % symmetic group
\newcommand{\sgn}{\operatorname{sgn}} % sign
\newcommand{\T}{\mathsf{T}} % use for tableaux
\newcommand{\U}{\mathsf{U}} % use for tableaux
\newcommand{\V}{\mathsf{V}} % use for tableaux
\newcommand{\TA}{\mathsf{A}} % use for tableaux
\newcommand{\TB}{\mathsf{B}} % use for tableaux
\newcommand{\TC}{\mathsf{C}} % use for tableaux
\newcommand{\TS}{\mathsf{S}} % use for tableaux
\newcommand{\shape}{\operatorname{shape}} % shape of a tableau
\newcommand{\col}[2]{\genfrac{}{}{0pt}{1}{#1}{#2}} % column of bitableau
\newcommand{\ov}{\overline} % shorthand for a bar on a symbol
\newcommand{\dd}{\partial} % use for diagram basis; e.g. d(V_l)
\newcommand{\X}{\mathcal{X}} % use for the Gelfand-Tsetlin subalgebra
\newcommand{\JM}{\mathcal{J}} % a subalgebra of the GT-subalgebra
\newcommand{\Std}{\operatorname{Std}} % set of standard tableaux
\newcommand{\StdB}{\operatorname{StdB}} % set of standard bitableaux
\newcommand{\Orb}{\mathcal{O}} % use for orbits
\newcommand{\OS}{\ov{\Orb}} % use for orbit sums
\newcommand{\OSS}{\ov{\OS}} % use for double bar orbit sums
\newcommand{\OR}{\mathscr{R}} % orbit representatives
\newcommand{\Stab}{\operatorname{Stab}} % stabilizer
\newcommand{\rev}{\operatorname{rev}} % reverse of a cycle
\newcommand{\A}{\mathcal{A}} % the algebra
\newcommand{\fraka}{\mathfrak{a}} % Young symmetrizer
\newcommand{\frakb}{\mathfrak{b}} % Young symmetrizer
\newcommand{\frakc}{\varphi} % canonical basis
\newcommand{\yy}{\mathsf{y}} % Young symmetrizer; scaled
\newcommand{\idem}{\varepsilon} % primitive central idem in symm gp
\newcommand{\cA}{\mathcal{A}} % group algebra of symmetric group
\newcommand{\Tab}{\operatorname{Tab}} % trails in branching graph from source
\newcommand{\BG}{\mathbf{B}} % branching graph
\newcommand{\bb}{\varnothing} % the unique element of \Irr(0) 
\newcommand{\res}{\operatorname{res}} % restriction
\newcommand{\Irr}{\operatorname{Irr}} % irreps
\newcommand{\Wt}{\operatorname{Wt}} % possible content vectors for an irrep
\newcommand{\trace}{\operatorname{trace}} % the trace
\newcommand{\type}{\operatorname{type}} % type = generalized shape
\newcommand{\gen}[1]{\langle #1 \rangle} % use for generating sets
\newcommand{\parm}{\updelta} % Brauer algebra parameter
\newcommand{\sep}{\,|\,} % separator for two partitions - used in tables 
\newcommand{\covered}{\lessdot}
\newcommand{\qand}{\quad\hbox{and}\quad}
\newcommand{\subgroup}{\mathrel{<}}
\newcommand{\normsubgroup}{\mathrel{\unlhd}}
\swapnumbers %% put numbers in front of proclamations
\newtheorem{thm}{Theorem}[section]
\newtheorem*{thm*}{Theorem}
\newtheorem{lem}[thm]{Lemma}
\newtheorem*{lem*}{Lemma}
\newtheorem{prop}[thm]{Proposition}
\newtheorem*{prop*}{Proposition}
\newtheorem{cor}[thm]{Corollary}
\newtheorem*{cor*}{Corollary}
\newtheorem{conj}[thm]{Conjecture}
\newtheorem*{conj*}{Conjecture}

\theoremstyle{definition}
\newtheorem{defn}[thm]{Definition}
\newtheorem*{defn*}{Definition}
\newtheorem{example}[thm]{Example}
\newtheorem*{example*}{Example}
\newtheorem{examples}[thm]{Examples}
\newtheorem*{examples*}{Examples}
\newtheorem{alg}[thm]{Algorithm}
\newtheorem*{alg*}{Algorithm}
%\theoremstyle{remark}
\newtheorem{rmk}[thm]{Remark}
\newtheorem*{rmk*}{Remark}
\newtheorem{rmks}[thm]{Remarks}
\newtheorem*{rmks*}{Remarks}

%%%%%%%%%%%%%%%%%%%%%%%%%%%%%%%%%%%%%%%%%%%%%%%%%%%%%%%%%%%%%%%%%%%
\numberwithin{equation}{section} 
%% The following avoids conflict between numbers of proclamations 
%% and numbers of equations
\renewcommand{\theequation}{\thesection\alph{equation}} 
%%%%%%%%%%%%%%%%%%%%%%%%%%%%%%%%%%%%%%%%%%%%%%%%%%%%%%%%%%%%%%%%%%%
\parskip = 2pt
\allowdisplaybreaks
\renewcommand{\labelenumi}{(\theenumi)} % use round brackets
\renewcommand{\theenumi}{\alph{enumi}} % use alphabetic enumerations


\pagestyle{plain} % suppress the running head - for working document

\title[Abstract Algebra]{Abstract Algebra}
\author{George H. Seelinger (inspired from class by Abramenko)}
\date{Fall 2016}
\begin{document}
\maketitle
\section{Lecture 1: Normalizers and Centralizers}
\begin{defn}
    For $A \subset G$, we set $N_G(A) := \{g \in G | gAg^{-1} = A\}$ and
    $C_G(A) := \{g \in G | gag^{-1} = a, \forall a \in A\}$. 
\end{defn}
Note that $C_G(A) \subset N_G(A)$.
\begin{rmk}
    If $A \subgroup G$, then $A \normsubgroup N_G(A)$. In fact, $N_G(A)$ is the
    largest subgroup of $G$ in which $A$ is normal.
\end{rmk}
    The first part of this remark follows from the definition of $N_G(A)$. The
    second part follows by assuming that $N_G(A)$ is not the largest and then
    showing the ``largest'' is contained in $N_G(A)$.
\begin{rmk}
    If $A \subgroup G$, then $A \normsubgroup G$ if $N_G(A) = G$.
\end{rmk}
    This follows from the second part of the remark above.
    Now for some remakrs about centralizers.
    \begin{rmk}
        If $A \subgroup G$, then $A$ is abelian if and only if $A \subset
        C_G(A)$. Furthermore, if $A \subgroup Z(G)$, then $A \normsubgroup G$.
        This follows using basic commutativity arguments.
    \end{rmk}
    \begin{example}
        $Z(S_n) = \{ id \}$ if $n \geq 3$. To prove this, prove $\sigma \in S_n, \sigma \neq id \implies \sigma \notin Z(S_n)$. 

        $\sigma \neq id \implies \exists i,j \in \{1, \ldots, n\}$ where $i
        \neq j$ such that $\sigma(i) = j$. Now, choose $k \in \{1, \ldots, n\}$
        such that $k \neq i,j$. Now, let $\pi = \sigma(i,k)\sigma^{-1} =
        (\sigma(i), \sigma(j)) = (j, \sigma(k))$. So now, $\pi(j) = \sigma(k)
        \neq \sigma(i) = j$. However, $(ik)(j)=j$ since $j \neq i,k$. This
        means $\pi \neq (ik)$, but if $\sigma$ were in the center, then it
        would have. So, we have that $\sigma \notin C_{S_n}( (ik) ) \implies
        \sigma \notin Z(S_n)$.
    \end{example}
    \begin{lem}
        Let $G$ be a group and $H \subgroup Z(G)$. Note this means $H
        \normsubgroup G$. If $G/H$ is cyclic, then $G$ is abelian.
    \end{lem}
    \begin{proof}
        Since $G/H$ is cyclic, this means $\exists g \in G$ such that
        $G/H=\langle gH \rangle$. Now, let $x,y \in G$. Then, there are $n,m
        \in \Z$ such that $xH = (gH)^n = g^nH$ and, similarly, $yH = (gH)^m =
        g^mH$. This means $\exists h,h' \in H$ such that $x=g^nh$ and
        $y=g^mh'$. Thus, we get $xy = g^nhg^mh' = g^ng^mhh' = g^{n+m}hh'$ and
        $yx = g^mh'g^nh = g^mg^nh'h = g^{n+m}h'h = g^{n+m}hh'$. Thus, we have
        our commutativity, thereby showing $G$ is abelian.
    \end{proof}\begin{rmk}
        Note, simply having $H \subgroup Z(G)$ and $G/H$ abelian does not imply
        $G$ is abelian. For a counter-example, look at $Q_8 = \{\pm 1, \pm i,
        \pm j, \pm k\}$, the standard quaternion group. It is easy to compute
        that $Z(Q_8) = \{\pm 1\}$. Now, $Q_8/Z(Q_8) \cong \Z_2 \times \Z_2$,
        the Klien-4 group, which is abelian. However, $Q_8$ is not abelian!
    \end{rmk}
    \begin{defn}
        Let $A,B \subgroup G$, then \begin{itemize}
            \item $A^{-1} = \{a^{-1} | a \in A\}$.
            \item $AB = \{ab | a \in A, b \in B\}$.
        \end{itemize}
    \end{defn}
    \begin{lem}
        Let $G$ be a group with $A,B \subgroup G$. Then,
        \begin{enumerate}
            \item $AB \subgroup G$ if and only if $AB = BA$.
            \item If $A \subgroup N_G(B)$ or $B \subgroup N_G(A)$, then $AB =
                BA$ and hence $AB \subgroup G$ by 1. 
        \end{enumerate}
    \end{lem}
    \begin{proof}
        ($\Rightarrow$) Let $AB \subgroup G$. Then, $AB = (AB)^{-1} =
        B^{-1}A^{-1} = BA$.

        ($\Leftarrow$) Let $AB = BA$. Then, we simply must show that $AB$ has
        group properties.\begin{enumerate}
            \item $e \in A, e \in B \implies e = ee \in AB$.
            \item This is the same as above. $(AB)^{-1} = B^{-1}A^{-1} = B! = AB$.
            \item $(AB)(AB) = A(BA)B = A(AB)B = (AA)(BB) = AB$.
        \end{enumerate}. Thus, $AB \subgroup G$.

        Now, for the second part, let $A \subgroup N_G(B)$. Then, let $a \in A,
        b \in B$. We know $a,a^{-1} \in A \subset N_G(B)$. Then, we have $ab =
        aba^{-1}a = (aba^{-1})a$. However, we know $aba^{-1} \in B$ from the
        definition of $N_G(B)$. So, $AB \subset BA$. Similar logic shows $BA
        \subset AB$ and thus $AB = BA$.
    \end{proof}

    \section{Lecture 2}
    \begin{rmk}
        \begin{enumerate}
            \item If $A,B \subgroup G$ and if $A \normsubgroup G$ or $B \normsubgroup G$, then $AB \subgroup G$.
            \item If $A,B \normsubgroup G$, then $AB \normsubgroup G$.
        \end{enumerate}
    \end{rmk}
    \begin{proof}
        The first remark can be done by $(ab)(a'b') = aba'b^{-1}bb' = aa'b' = (aba'b^{-1})b' \in AB$ because $a$ and $ba'b^{-1}$ are in $A$ and $b \in B$. Also, $(ab)^{-1} = b^{-1}a^{-1} = (b^{1}a^{-1}b)b^{-1} \in AB$. The identity inclusion is obvious.
        The second remark is proven by, for $g \in G$, $gABg^{-1} = gAg^{-1}gBg^{-1} = AB$. 
    \end{proof}

    \subsection{Index Computations}

    \begin{lem}
        Let $A,B \subgroup G$ and $|A|, |B| < \infty$. Then, $|AB| = \frac{|A| \cdot |B|}{A \cap B}$.
    \end{lem}
    \begin{proof}
        Consider $A/(A \cap B)$. Take a complete set of coset representatives $A' = \{a_1, \ldots, a_n\}$. Then, $n = |A/(A \cap B)| = [A : A \cap B] = \frac{|A|}{|A \cap B|}$ by Lagrange's Theorem.

        Now, consider $f: A' \times B \to AB$ defined by $(a_i,b) \to a_ib$. If
        we show that $f$ is bijective, then we will have that $|A'| \cdot |B| =
        |A' \times B| = |AB|$, which will allow us to finish the proof.
        To show $f$ is injective, let us take \begin{align*}
            a_ib = a_jb' (1 \leq i,j \leq n; b,b' \in B) & \ \implies \ a_j^{-1} a_i = b'b^{-1} \in A\cap B\\
            \ & \ \implies \ a_i(A \cap B) = a_j(A \cap B) \\
            \ & \ \implies \ a_i = a_j \\
            a_ib = a_jb' & \ \implies b = b' \\
            \ & \ \implies (a_i, b) = (a_j, b')
        \end{align*}

        To show $f$ is surjective, we let $a \in A, b \in B$. Then $\exists i
        (1 \leq i \leq n)$ such that $a=a_ix$ with $x \in A \cup B$. This
        implies that $xb \in B$ so $ab = a_ixb = f( (a_i,xb) )$.

        Thus, we have that $f$ is bijective, and so we conclude that $n \cdot
        |B| = \frac{|A|}{|A \cap B|} \cdot |B|$ giving us our result when we
        divide by $|B|$.
    \end{proof}
    \begin{example}
        (a) $G = \S_3, A = \langle (12) \rangle, B = \langle (13) \rangle
        \subgroup G$. Then, $A \cap B = \{ id \} \implies |AB| = 2 \cdot 2 = 4$
        which does not divide $6 = |\S_3|$. Thus, $AB$ is not a subgroup of
        $\S_3$. 
        (b) Let $G=\S_4, A = S_3 \to S_4, B = \langle (1234) \rangle$. Then, $A \cap B = \{id\} \implies |AB| = |A||B| = 6 \cdot 4 = 24 = |\S_4|$. This means that $AB = \S_4$ but note that $A \not\subset N_G(B)$ and $B \not\subset N_G(A)$. 
    \end{example}
    \begin{lem}
        (Lemma 1.1.9) Let $G$ be a group and $A,B \subgroup G$. Then
        \begin{enumerate}[label=\Alph*]
            \item If $A \subgroup B$, then $[G:A] = [G:B][B:A]$
            \item $[A: A \cup B] \leq [G:B]$
            \item $[G: A \cup B] \leq [G:A][G:B]$
        \end{enumerate}
    \end{lem}
    \begin{proof}
       For the first part, fix a set of coset representatives (using the Axiom of Choice in the infinite case) for $G/A$. Then, given a coset $gA$, we can TODO 
\end{proof}
\subsection{The Isomorphism Theorems}
\begin{thm}
    (1.1.10 First Isomorphism Theorem) If $\phi: G \to H$ is a group homomorphism, then $G/\ker \phi \cong \phi(G) \subgroup H$.
\end{thm}
\begin{proof}
    Set $N = \ker \phi \normsubgroup G$. Then, consider $\widetilde{\phi}: G/N \to \phi(G)$ defined by $\widetilde{\phi}(gN) = \phi(g)$. First, we check that $\widetilde{\phi}$ is well-defined.

    Assume $gN = g'N, g,g' \in G$. Then, there exists $n \ in N$ such that $g' = gn$. This tells us that $\phi(g') = \phi(gn) = \phi(g)\phi(n) = \phi(g)e = \phi(g)$. Now, we show that $\widetilde{\phi}$ is a group homomorphism.

    Check $\widetilde{\phi}( (gN)(g'N) ) = \widetilde{\phi}(gg'N) = \phi(gg') = \phi(g)\phi(g') = \widetilde{\phi}(gN) \widetilde{\phi}(g'N)$.

    We know $\widetilde{\phi}$ is surjective by definition. 

    To show $\widetilde{\phi}$ is injective, it suffices to show that $\ker \widetilde{\phi} = \{\widetilde{e} = eN\}$. If $\widetilde{\phi}(gN) = \phi(g) = e \in H$, then $g \in \ker \phi = N$ and therefore $gN = N = \widetilde{e}$
\end{proof}
\begin{cor}
    (1.1.11 Second Isomorphism Theorem) If $A,B \subgroup G$ and $A \subgroup N_G(B)$, then $AB/B \cong A/A\cap B$.
\end{cor}
\begin{proof}

\end{proof}<++>
\end{document}
