\documentclass[master.tex]{subfiles}

\setcounter{section}{1}

\begin{document}
\section{Ring Theory}

\newcommand{\F}{\mathbb{F}}
\newtheorem*{notation}{Notation}
% Lecture 10/27/2016

\begin{defn*}
  A \emph{ring} is set \(R\) together with two binary operations \(+\) and \(\cdot\) satisfying
  \begin{enumerate}[label=(\roman*)]
  \item \((R,+)\) is an abelian group (denote the additive identity by \(0\))
  \item \(\cdot\) is associative, \((xy)z=x(yz)\) for all \(x,y \in R\)
  \item There exists a multiplicative identity (denoted \(1 \in R\)).
  \item Distribution laws hold:
    \begin{align*}
      x(y+z) &= xy + xz\\
      (x+y)z &= xz+yz
    \end{align*}
  \end{enumerate}
\end{defn*}

\begin{defn*}
  A \emph{commutative ring} is a ring with the additional property that \(xy=yx\) for all \(x,y \in R\).
\end{defn*}
Notice that if multiplication is commutative either distribution law implies the other. One of the most basic
observations one can make is that \(R\) is the trivial ring \(\iff 1=0\).

\begin{example*}
  A few familiar rings: \[\Z,\Z_n,\Q,\R,\C\] A non-commutative ring:
  \[M_n(\F), n \ge 2\] or more generally take any ring \(R\).
\end{example*}

\begin{defn*}
  A \emph{skew field (or division ring)} is a ring \(R\) such that \(R \neq \{0\}\) and \(R \setminus \{0\}\) is a group
  under multiplication.
\end{defn*}

\begin{defn*}
  A \emph{field} is a commutative skew field.
\end{defn*}

\begin{example*}
  An example of a skew field is
  \[\H = \R + \R i + \R j + \R k\]
\end{example*}

One may form direct products of rings in the usual manner.

\begin{defn*}
  A \emph{ring homomorphism} is map \(\varphi \colon R \to S\) between rings \(R\) and \(S\) satisfying
  \begin{enumerate}[label=(\arabic*)]
  \item \(\varphi(x+y)=\varphi(x)+\varphi(y)\)
  \item \(\varphi(xy)=\varphi(x)\varphi(y)\)
  \item \(\varphi(1_R)=1_S\)
  \end{enumerate}
\end{defn*}

In keeping with our demand that all of rings have unity, we also force our homomorphisms to respect the unital
structure. In particular our ring homomorphisms are morphisms in the category of commutative rings.

\begin{example*}
  For us a map \funcdeclaration{\varphi}{R}{R \times R}{r}{(r,0)} is not a ring homomorphism as
  \(1_{R \times R} = (1,1)\) while \(\varphi(1)=(1,0)\).
\end{example*}

\begin{defn*}
  A subset \(S\) of a ring \(R\) is a \emph{subring} of \(R\) if
  \begin{enumerate}[label=(\arabic*)]
  \item \(S\) is closed under both operations
  \item \(1_R \in S\)
  \end{enumerate}
\end{defn*}

\begin{example*}
  Under this definition \(R \times \{0\}\) is not a subring of \(R \times R\) as \(1_R \not \in R \times R\) as long as
  \(R\neq\{0\}\).
\end{example*}

\begin{defn*}
  An element \(x \in R\) is said to be a \emph{zero divisor} of \(R\) if there exists an element \(y \in R\) such that
  \(y \neq 0\) and
  \[xy = 0 \qquad \text{ or } \qquad yx = 0.\]
\end{defn*}

\begin{defn*}
  The commutative ring \(R\) is an \emph{integral domain} if its only zero divisor is \(0\).
\end{defn*}

\begin{example*}
  \(\Z_n\) integral domain \(\iff n\) is prime
\end{example*}

\begin{prop*}
  \(R\) finite integral domain \(\implies\) \(R\) is field.
\end{prop*}

\begin{defn*}
  A \(x \in R\) is a \emph{unit} if \(xy=1\) and \(yx=1\).
\end{defn*}

\begin{defn*}
  The group of units \(R^\times\) of a ring is
  \[R^\times = \{x \in R \mid x\text{ is a unit}\}.\]
\end{defn*}

\begin{example*}
  \begin{align*}
    \Z^\times &= \{-1,1\}\\
    \Z_n^\times &= \{\bar{a} \mid (a,n)=1\}
  \end{align*}
\end{example*}

\begin{prop*}
  If \(\varphi \colon R \to S\) is a homomorphism and \(x \in R^{\times}\) then \(\varphi(x) \in S^\times\). In
  particular this means \(\varphi(R^\times) \le S^\times\).
\end{prop*}

Even if \(\varphi \colon R \twoheadrightarrow S\) the image \(\varphi(R^\times)\) might not equal \(S^\times\).

\begin{example*}
  Consider the surjective function
  \funcdeclaration{\varphi}{\Z}{\Z_n}{a}{\bar{a}}
  but \(|\Z_n^\times|=\varphi(n)>2\) if \(n \ge 7\)
\end{example*}

\begin{notation}
  From this point onwards \(R\) denotes a \textbf{commutative ring}.
\end{notation}

\begin{defn*}
  A non-empty subset \(I \subset R\) is an \emph{ideal} if 
  \begin{enumerate}[label=(\roman*)]
  \item  \(x,y \in I \implies x+y \in I \)
  \item \(x \in I, r \in R \implies rx \in I\)
  \end{enumerate}
\end{defn*}
Condition one may be rephrased as \((I,+)\) forms an abelian group.
\begin{prop*}
  An ideal \(I=R \iff 1 \in R \iff I \cap R^\times \neq \emptyset\)
\end{prop*}

\begin{notation}
  If \(I\) is an ideal of \(R\) we denote it as \(I \unlhd R\).
\end{notation}

\begin{prop}
  A ring homomorphism \(\varphi \colon R \to S\) if \(I \unlhd R\), \(J \unlhd S\) then
  \[\varphi(I) \unlhd S \iff \varphi \text{ is surjective }\]
  while
  \[\varphi^{-1} \unlhd R \text{ is always true.}\]
\end{prop}

\begin{defn*}
  Let \(I,J \unlhd R\) then we define
  \begin{align*}
    I+J &:= \{x+y \mid x \in I, y \in J\} \unlhd R\\
    IJ  &:= \left\{\sum_{l=0}^nx_l y_l \mid n \in \N, x_l \in I, y_l \in J\right\}.
  \end{align*}
\end{defn*}
We have defined \(IJ\) in the above manner to force \(IJ\) to be an abelian group, and hence an ideal.

\begin{defn*}
  Given an ideal \(I \unlhd R\) we may form the \emph{quotient ring} denote \(R/I\) as follows
  \[R/I = \{a+I\mid a \in R\}.\]
  This forms a ring under addition and multiplication by representatives.
\end{defn*}
\end{document}

%%% Local Variables:
%%% mode: latex

%%% TeX-master: "master.tex"
%%% End:
