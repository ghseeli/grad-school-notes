\documentclass[master.tex]{subfiles}

\begin{document}
\section{Ring Theory}

\newcommand{\F}{\mathbb{F}}

% Lecture 10/27/2016

\begin{defn*}
  A \emph{ring} is set \(R\) together with two binary operations \(+\) and \(\cdot\) satisfying
  \begin{enumerate}[label=(\roman*)]
  \item \((R,+)\) is an abelian group (denote the additive identity by \(0\))
  \item \(\cdot\) is associative, \((xy)z=x(yz)\) for all \(x,y \in R\)
  \item There exists a multiplicative identity (denoted \(1 \in R\)).
  \item Distribution laws hold:
    \begin{align*}
      x(y+z) &= xy + xz\\
      (x+y)z &= xz+yz
    \end{align*}
  \end{enumerate}
\end{defn*}

\begin{defn*}
  A \emph{commutative ring} is a ring with the additional property that \(xy=yx\) for all \(x,y \in R\).
\end{defn*}
Notice that if multiplication is commutative either distribution law implies the other. One of the most basic
observations one can make is
that \(R\) is the trivial ring \(\iff 1=0\).

\begin{example*}
  A few familiar rings: \[\Z,\Z_n,\Q,\R,\C\]
  A non-commutative ring:
  \[M_n(\F), n \ge 2\]
  or more generally take any ring \(R\).
\end{example*}

\begin{defn*}
  A \emph{skew field (or division ring)} is a ring \(R\) such that \(R \neq \{0\}\) and \(R \setminus \{0\}\) is a group
  under multiplication.
\end{defn*}

\begin{defn*}
  A \emph{field} is a commutative skew field.
\end{defn*}

\begin{example*}
  An example of a skew field is
  \[\H = \R + \R i + \R j + \R k\]
\end{example*}

\end{document}

%%% Local Variables:
%%% mode: latex
%%% TeX-master: "master.tex"
%%% End:
