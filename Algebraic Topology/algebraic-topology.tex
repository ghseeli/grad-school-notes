\documentclass[11pt,leqno,oneside]{amsart}
\usepackage[margin=1in]{geometry}

\usepackage{../notes}
\usepackage{xfrac}
\usepackage{enumitem}
\usepackage{tikz-cd}
\usetikzlibrary{cd}
\usepackage{hyperref}
\usepackage{mathtools}
\mathtoolsset{showonlyrefs}

\usepackage{datetime}
% an environment specifically for my dates, although it doesn't actually depend on datetime:
\newenvironment{dateenv}{
  \vspace{1em}
}{
  \vspace{1em}
}
% my date command, for convenient date adding
\newcommand{\mydate}[4]{
  \newdate{#1}{#2}{#3}{#4}
  % #1 is a unique string, #2 is day, #3 month, #4 year
  \begin{dateenv}
    \hfill\displaydate{#1}
  \end{dateenv}
}

\numberwithin{thm}{section}
\setlength\parindent{0pt}

\everymath{\displaystyle}

\newcommand{\minus}{\smallsetminus}
\renewcommand{\setminus}{\smallsetminus}
\renewcommand{\amalg}{\sqcup}
\newcommand{\homotopic}{\simeq}
\renewcommand{\epsilon}{\varepsilon}
\renewcommand{\d}{\partial}
%\newcommand{\oo}{\infty}
\newcommand{\transverse}{\pitchfork}
\newcommand{\into}{\hookrightarrow}
\newcommand{\onto}{\twoheadrightarrow}
\newcommand{\x}{\times}
\newcommand{\Map}{\text{Map}}
\newcommand{\id}{\text{id}}
\renewcommand{\bar}{\widebar}
\newcommand{\de}{\emph}

%%%%%%%%%%%%%% BEGIN CONTENT: %%%%%%%%%%%%%%

\title[Algebraic Topology]{Algebraic Topology MATH 7800}
\author{Chris Lloyd, Matthew Lancellotti}
\date{}
\begin{document}
\maketitle \newpage

\mydate{d1}{18}{1}{2017}

\section{Introduction}

Topology is floppy. Given just a topological space one has an infinite
degree of freedom. This makes even simple questions hard to answer, in
particular given manifolds \(M\) and \(N\) with different dimensions
\(m\) and \(n\) respectively, one may ask if they are
homeomorphic. Intuitively one would think that should not be
possible. However there do exist space filling curves, i.e,
\[I \twoheadrightarrow I \times I\]
where as always \(I=[0,1]\).

We will develop the tools to prove manifolds of different dimensions can
certainly never be homeomorphic. Another surface result assures there exists no continuous
retraction \(D^2 \to \d D\), which then immediately yields the
Brouwer Fixed Pointed Theorem (every map \(D^2 \to D^2\) admits a
fixed point).

In this course there will be two basic branches under consideration:
homotopy theory and homology. Basically, homotopy concerns itself with
continuous deformations between maps. While homology studies subspaces
(known as ``cycles'') and gap filling.

\section{Reference}

\begin{defn}
  $D^n$ denotes the $n$-dimensional closed unit disk.
\end{defn}

\section{Homotopy}

We now define homotopies in the space of maps between manifolds \(M\)
and \(N\) denoted
\[\Map(M,N)=\{f \colon M \to N \mid \text{$f$ is continuous}\}.\]

In this course, we follow the convention that maps are continuous
functions unless otherwise stated!

\begin{defn}
  A \emph{homotopy} from \(f \colon M \to N\) to \(g \colon M \to N\)
  is a map \(h\) such that \(h_0=f\) and \(h_1=g\). We say \(f\) and
  \(g\) are \emph{homotopic} if a homotopy between them exists.

  Often we write \(h_t(m)\) for \(h(m,t)\).
\end{defn}

When working with homotopies one often thinks of having the start and
end map already fixed. For maps to be homotopic they must be in the
same path component.

\begin{prop}
  Homotopy is an equivalence relation.
\end{prop}

We denote the homotopy equivalence class of \(f\) by \([f]\).

\begin{defn}
  A \emph{path} in a topological space $X$ is a map
  \[f \colon I \to X.\]
\end{defn}

By convention in this course, we always consider paths up to homotopy,
however we will make a restriction and only consider homotopies that
fix end points \((I,\{0,1\})\). Otherwise every path would collapse to
a point up to homotopy. This is called \emph{relative homotopy}. When
there are no restrictions placed one is working with \emph{free
  homotopy}.

Imagine two paths in \(\R \setminus (0,0)\) which agree on end points,
however each path is on opposite sides of the origin. It is clear that
there exists no end point fixing homotopy between the two paths.

\begin{defn}
  If one has two paths \(f,g\) with \(f(1)=g(0)\) we may define a new
  path
  \[f * g =
    \begin{cases}
      f(2t), & t \in [0,\frac{1}{2})\\
      g(2t - 1), & t \in [\frac{1}{2},1].
    \end{cases}
  \]

  This is called the \emph{concatenation} or \emph{path composition}
  of $f$ and $g$.

  This operation is well defined on homotopy classes, that is if
  \([f]=[f']\) and \([g]=[g']\) then \([f * g] = [f' * g']\). This
  operation may be generalized to \(*_{k}\) where \(k \in (0,1)\),
  simply replace \(\frac{1}{2}\) with \(k\). In this notation
  \(f * g = f *_{\frac{1}{2}} g\). This operation is associative.
\end{defn}

Given a topological space \(X\) with points \(P\), for any two points
\(x,y \in P\) we have a set of homotopy classes with end points \(x\)
and \(y\). These may be thought of as morphisms and the points as
objects in a category. We call this category the Fundamental Groupoid.

\mydate{d2}{20}{1}{2017}

\begin{defn}
  A \de{category} is a directed graph.  Each vertex is called an
  \de{object} and each directed edge is called a \de{morphism}.
\end{defn}
\begin{defn}
  In the category of \de{finite sets}, denoted $\text{FSet}$, the
  objects are the finite sets, each morphism from $A$ to $B$ is a
  function from $A$ to $B$.
\end{defn}
\begin{defn}
  For any field $K$, the category of \de{$K$-vector spaces}, denoted
  $\text{Vect}_K$, has objects that are $K$-vector spaces and
  morphisms that are $K$-linear maps.
\end{defn}
\begin{defn}
  A \de{groupoid} is a category where every morphism has an inverse.

  That is, if $a: x \to x'$, then there exists $a^{-1}:x' to x$
  s.t. $aa^{-1} = \id_{x'}$ and $a^{-1}a = \id_{x}$.

  The elements of the groupoid are the morphisms.  The operation is
  composition.
\end{defn}
\begin{prop}
  Every morphism in a groupoid is an isomorphism (by definition).
\end{prop}
\begin{defn}
  A \de{group} is a groupoid with one object.
\end{defn}
\begin{prop}
  If $A$ is an object in a groupoid, then $\Hom(A, A)$ is a group.
\end{prop}

note We see that a groupoid is the same definition of a group, but we
use a *partial* binary operation instead of a binary operation.

\begin{defn}
  The \de{fundamental groupoid} of $X$, denoted $\Pi_1 X$, is the
  category where the objects are the points in $X$ and the morphisms
  are the paths (up to endpoint fixing homotopy) between those points.
\end{defn}
\begin{defn}
  The \de{fundamental group of $X$ at the basepoint $x_0$}, denoted
  $\pi_1(X, x_0)$, is the subgroupoid of the fundamental groupoid
  where there is exactly one object, $x_0$, and the morphisms are the
  morphisms of the fundamental groupoid that have source and target
  $x_0$.  Another wording is to say that $\pi_1(X, x_0)$ is the full
  subcategory of $\Pi_1 X$ spanned by $x_0$.

  The elements of the group are the morphisms.  The operation is
  composition.
\end{defn}
\begin{defn}
  the \de{inverse path} of a path $p$ is $p^{-1}(s) := p(1-s)$.
  (note: This is *not* the same as the function inverse of $p$)
\end{defn}
\begin{prop}
  If $p$ is a path, then
  $[p^{-1}*p] = [id_{x}] = \text{the function where $s \mapsto x$ for
    all $s \in [0,1]$}$.
\end{prop}

\begin{thm}[Hatcher 1.5]
  Let $X$ be a metric space and $x,x' \in X$.  If there exists
  $p: x \to x'$ and $[q] \in \pi_1(X,x')$, then
  $p^{-1}qp \in \Pi_1(X,x)$ is an isomorphism of fundamental groups.
\end{thm}
\begin{thm}
  If $X$ is path connected, then for any $x_a, x_b \in X$,
  $$\Pi_1(X, x_a) \isom \Pi_1(X, x_b).$$
  notation If $X$ is path connected, we can denote the fundamental
  group with $\pi_1(X)$ or $\pi_1 X$.
\end{thm}
\begin{example}
  pretty picture

  This isomorphism is not canonical
  

  Note that $\pi_1 X$ is only well defined up to inner automorphism.
  ((Chris, what does this mean?))
\end{example}
\begin{defn}
  Given metric spaces $X, Y$ and a map $f: X \to Y$, then the
  \de{fundamental functor} induced by $f$,
  $$f_* = \Pi_1 f : \Pi_1 X \to \Pi_1 Y$$ maps each object $x \in X$
  to $f(x) \in Y$ and each path $[p] \in \Pi_1 X$ to
  $f([p]) = [f(p)] \in \Pi_1 Y$.
\end{defn}
\begin{prop}
  The above is well defined because
  \begin{itemize}
  \item If $[p]=[q]$, then $[f(p)] = [f(q)]$ because $h.f$ gives a
    homotopy.  ((What the heck is h?))
  \item $(f_*p)*(f_*q) = f_*(p*q)$
  \end{itemize}
\end{prop}

\begin{prop}
  Given a fundamental group $\Pi_1(X, x)$, then we can restrict $f_*$
  to the functor $$f_* : \Pi_1(X, x) to \Pi_1(Y, f(x)).$$

  picture $x$ (loop) to $f(x)$ (loop)

  is a group homomorphism
\end{prop}
\begin{prop}
  If $X,Y,Z$ are metric spaces and $f: X \to Y$, $g: Y \to Z$ are
  maps, then ${(f.g)}_* = f_*.g_*$.
\end{prop}

\begin{example}
  $\Pi_1(\R^n, x) = [*]$

  in $\Pi_1 \R^n$, there exists a unique path between any two points,
  up to homotopy.

  pf $p,q: I \to \R^n$

  $p(0) = q(0)$ $p(1) = q(1)$

  $h_t = tp + (1-t)q$

  $h_0 = q$ $h_1 = p$

  $B_n, I_n$ all have $\Pi_1 = [*]$
\end{example}

\begin{thm}[Hatcher 1.6]
  A space $X$ is \de{simply connected} or \de{1-connected} iff its
  fundamental group is trivial.
\end{thm}
\begin{example}
  $\R^2 \minus (0,0) \isom \C \minus \{0\}$ is NOT simply connected.
  $\Pi_1(\C \minus \{0\}) = \Z$
\end{example}
\begin{example}
  pretty picture with yellow and blue dots

  $A$ and $B$ can't be homotopic because $\log(A)$ and $\log(B)$ have
  different end points.
\end{example}

\begin{defn}
  A \emph{covering space} of a space \(X\) is a space \(\tilde{X}\)
  and map \(p \colon \tilde{X} \to X\). Where there eixsts a cover
  \(\{U_\alpha\}\) of \(X\) such that \(p^{-1}(U_\alpha)\) is a
  disjoint union of open sets in \(\tilde{X}\) each of which has image
  under \(p\) homeomorphic to \(U_\alpha\).
\end{defn}

Think of the stack of records theorem from Guillemin and Pollack. See
page 59 in Hatcher for the covering space of two loops as discussed in
class. We now discuss homotopy lifting.

\begin{prop}
  Given a covering space \((\tilde{X},p)\) of \(X\) and a homotopy
  \(h_t \colon Y \to X\), and some map
  \(\tilde{f}_0 \colon Y \to \tilde{X}\), then there exists a unique
  homotopy \(f_t \colon Y \to X\) that lifts \(f_0\), that is
  \begin{center}
    \begin{tikzcd}
      & & \tilde{X}\\
      Y \times \{0\} \arrow[r,hook] \arrow[rru,"\tilde{f}_0", bend
      left]& Y \times I \arrow[r,"f_t",swap] \arrow[ru,dashed,"\tilde{f}_t"]& X \arrow[u,"p"]
    \end{tikzcd}
  \end{center}
\end{prop}

\end{document}

%%% Local Variables:
%%% mode: latex
%%% TeX-master: "algebraic-topology.tex"
%%% End:
