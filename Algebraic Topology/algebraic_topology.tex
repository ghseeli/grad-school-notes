\documentclass[11pt,leqno,oneside]{amsart}
\usepackage[margin=1in]{geometry}

\usepackage{../notes}
\usepackage{xfrac}
\usepackage{enumitem}
\usepackage{tikz-cd}
\usepackage{mathtools}

\numberwithin{thm}{subsection}

%%%%%%%%%%%%%% BEGIN CONTENT: %%%%%%%%%%%%%%

\title[Algebraic Topology]{Algebraic Topology - MATH 7800}
\author{Chris Lloyd, Matthew Lancellotti}
\date{}
\begin{document}
\maketitle \newpage

\section{Introduction}

Topology is floppy. Given just a topological space one has an infinite
degree of freedom. This makes even simple questions hard to answer, in
particular given manifolds \(M\) and \(N\) with different dimensions
\(m\) and \(n\) respectively, one may ask if they are
homeomorphic. Intuitively one would think that should not be
possible. However there do exist space filling curves, i.e,
\[I \twoheadrightarrow I \times I\]
where as always \(I=[0,1]\). 

We will develop the tools to prove manifolds of different dimensions can
certainly never be homeomorphic. Another surface result assures there exists no continuous
retraction \(D^2 \to \partial D\), which then immediately yields the
Brouwer Fixed Pointed Theorem (every map \(D^2 \to D^2\) admits a
fixed point).

In this course there will be two basic branches under consideration,
homotopy theory, and homology. Basically homotopy concerns itself with
continuous deformations between maps. While homology studies subspaces
(known as ``cycles'') and gap filling.


\end{document}

%%% Local Variables:
%%% mode: latex
%%% TeX-master: "algebraic_topology.tex"
%%% End:
