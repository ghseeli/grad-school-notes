\documentclass[11pt,leqno,oneside]{amsart}
\usepackage[margin=1in]{geometry}

\usepackage{../notes}
\usepackage{xfrac}
\usepackage{enumitem}
\usepackage{tikz-cd}
\usetikzlibrary{cd}
\usepackage{hyperref}
\usepackage{mathtools}
\mathtoolsset{showonlyrefs}

\usepackage{datetime}
% an environment specifically for my dates, although it doesn't actually depend on datetime:
\newenvironment{dateenv}{
  \vspace{1em}
}{
  \vspace{1em}
}
% my date command, for convenient date adding
\newcommand{\mydate}[4]{
  \newdate{#1}{#2}{#3}{#4}
  % #1 is a unique string, #2 is day, #3 month, #4 year
  \begin{dateenv}
    \hfill\displaydate{#1}
  \end{dateenv}
}

\numberwithin{thm}{section}
\setlength\parindent{0pt}

\everymath{\displaystyle}

\newcommand{\minus}{\smallsetminus}
\renewcommand{\setminus}{\smallsetminus}
\renewcommand{\amalg}{\sqcup}
\newcommand{\homotopic}{\simeq}
\renewcommand{\epsilon}{\varepsilon}
\renewcommand{\d}{\partial}
\newcommand{\oo}{\infty}
\newcommand{\transverse}{\pitchfork}
\newcommand{\into}{\hookrightarrow}
\newcommand{\onto}{\twoheadrightarrow}
\newcommand{\x}{\times}
\newcommand{\Map}{\text{Map}}
\renewcommand{\bar}{\widebar}

%%%%%%%%%%%%%% BEGIN CONTENT: %%%%%%%%%%%%%%

\title[Algebraic Topology]{Algebraic Topology - MATH 7800}
\author{Chris Lloyd, Matthew Lancellotti}
\date{}
\begin{document}
\maketitle \newpage

\mydate{d1}{18}{1}{2017}

\section{Introduction}

Topology is floppy. Given just a topological space one has an infinite
degree of freedom. This makes even simple questions hard to answer, in
particular given manifolds \(M\) and \(N\) with different dimensions
\(m\) and \(n\) respectively, one may ask if they are
homeomorphic. Intuitively one would think that should not be
possible. However there do exist space filling curves, i.e,
\[I \twoheadrightarrow I \times I\]
where as always \(I=[0,1]\).

We will develop the tools to prove manifolds of different dimensions can
certainly never be homeomorphic. Another surface result assures there exists no continuous
retraction \(D^2 \to \d D\), which then immediately yields the
Brouwer Fixed Pointed Theorem (every map \(D^2 \to D^2\) admits a
fixed point).

In this course there will be two basic branches under consideration:
homotopy theory and homology. Basically, homotopy concerns itself with
continuous deformations between maps. While homology studies subspaces
(known as ``cycles'') and gap filling.

\section{Reference}

\begin{defn}
  $D^n$ denotes the $n$-dimensional closed unit disk.
\end{defn}

\section{Homotopy}

We now define homotopies in the space of maps between manifolds \(M\) and
\(N\) denoted
\[\Map(M,N)=\{f: M \to N \mid \text{$f$ is continuous}\}.\]

In this course, we follow the convention that maps are continuous functions unless otherwise stated!

\begin{defn}
  A \emph{homotopy} from \(f: M \to N\) to \(g: M \to N\)
  is a map \(h\) such that \(h_0=f\) and \(h_1=g\). We say \(f\) and
  \(g\) are \emph{homotopic} if a homotopy between them exists.

  Often we write \(h_t(m)\) for \(h(m,t)\).
\end{defn}

When working with homotopies one often thinks of having the start and
end map already fixed. For maps to be homotopic they must be in the
same path component.

\begin{prop}
  Homotopy is an equivalence relation.
\end{prop}

We denote the homotopy equivalence class of \(f\) by \([f]\).

\begin{defn}
  A \emph{path} in a topological space $X$ is a map
  \[f: I \to X.\]
\end{defn}

By convention in this course, we always consider paths up to homotopy, however we will make a
restriction and only consider homotopies that fix end
points \((I,\{0,1\})\). Otherwise every path would collapse to a point up to
homotopy. This is called \emph{relative homotopy}. When there are no
restrictions placed one is working with \emph{free homotopy}.

Imagine two paths in \(\R \setminus (0,0)\) which agree on end points,
however each path is on opposite sides of the origin. It is clear that
there exists no end point fixing homotopy between the two paths.

\begin{defn}
  If one has two paths \(f,g\) with \(f(1)=g(0)\) we may define a new
  path
  \[f * g =
    \begin{cases}
      f(2t), & t \in [0,\frac{1}{2})\\
      g(2t - 1), & t \in [\frac{1}{2},1].
    \end{cases}
  \]

  This is called the \emph{concatenation} or \emph{path composition} of $f$ and $g$.

  This operation is well defined on homotopy classes, that is if
  \([f]=[f']\) and \([g]=[g']\) then \([f * g] = [f' * g']\). This
  operation may be generalized to \(*_{k}\) where \(k \in (0,1)\),
  simply replace \(\frac{1}{2}\) with \(k\). In this notation \(f * g =
  f *_{\frac{1}{2}} g\). This operation is associative.
\end{defn}

Given a topological space \(X\) with points \(P\), for any two points
\(x,y \in P\) we have a set of homotopy classes with end points \(x\)
and \(y\). These may be thought of as morphisms and the points as
objects in a category. We call this category the Fundamental Groupoid.

\end{document}

%%% Local Variables:
%%% mode: latex
%%% TeX-master: "algebraic_topology.tex"
%%% End:
